\chapter{Dados Abertos}

De acordo com a definição da Open Definition: “dado aberto é um dado que pode ser livremente utilizado, reutilizado e redistribuído por qualquer um”. A definição completa pode ser encontrada em \cite{open}, mas abaixo serão listados alguns pontos importantes para o contexto em questão:
\begin{itemize}
\item Disponibilidade e Acesso: os dados devem estar disponíveis como um todo e sob custo não maior que um custo razoável de reprodução, preferencialmente possíveis de serem baixados pela internet. Os dados devem também estar disponíveis de uma forma conveniente e modificável.
\item Reutilização e Redistribuição: os dados devem ser fornecidos sob termos que permitam a reutilização e a redistribuição, inclusive a combinação com outros conjuntos de dados.
\item Participação Universal: todos devem ser capazes de usar, reutilizar e redistribuir - não deve haver discriminação contra áreas de atuação ou contra pessoas ou grupos. Por exemplo, restrições de uso ‘não-comercial’ que impediriam o uso ‘comercial’, ou restrições de uso para certos fins (ex.: somente educativos) excluem determinados dados do conceito de ‘abertos’.
\end{itemize}

Os dados abertos podem vir de qualquer fonte. Existem dados abertos na ciência, dados de empresas privadas e o mais importante para esse trabalho, Dados Abertos no governo.

Dados abertos governamentais são dados produzidos pelo governo e colocados à disposição das pessoas de forma a tornar possível não apenas sua leitura e acompanhamento, mas também sua reutilização em novos projetos, sítios e aplicativos; seu cruzamento com outros dados de diferentes fontes; e sua disposição em visualizações interessantes e esclarecedoras. \cite{manual}.

Apenas o fato de uma organização publicar seus dados na Web não os tornam abertos. Para que eles sejam considerados Dados Abertos, a organização deve respeitar uma série de princípios, onde os dados devem ser:
\begin{table}[H]
\begin{center}
    \begin{tabular}{ | l | p{11cm} | }
    \hline
   Completos & Todos os dados públicos são disponibilizados. Dados públicos são dados que não se submetem à limitações válidas de privacidade, segurança ou privilégio. \\ \hline
    Primários & Os dados são coletados na sua fonte, com o maior nível possível de granularidade, não estando em formas agregadas ou modificadas. \\ \hline
   Atualizados & Os dados são disponibilizados tão rápido quanto seja necessário para preservar seu valor. \\ \hline
   Acessíveis & Os dados estão disponíveis para o maior escopo possível de usuários e para o maior escopo possível de finalidades. \\ \hline
Não-discriminatórios & Os dados estão disponíveis para todos, sem necessidade de registro. \\ \hline
Não-proprietários & Os dados são disponibilizados num formato do qual nenhuma entidade tem controle exclusivo.\\ \hline
Livres de licenças & Os dados não estão sujeitos a nenhuma forma de direito autoral, patente, propriedade intelectual ou segredo industrial. Restrições razoáveis de privacidade, segurança e privilégio podem ser permitidas.\\ \hline
   \end{tabular}
    \caption{Oito principios dos dados aberto governamentais.}
    \label{tab-principios}
\end{center}
\end{table}
Todos os oito princípios citados na tabela acima foram definidos num evento que reuniu mais de 30 ativistas em prol da abertura dos governos. O evento foi realizado em Sebastopol, na Califórnia, e tinha o objetivo de “desenvolver um entendimento mais robusto de porque dados governamentais abertos são essenciais para a democracia”.

Analisando os princípios elencados na tabela \ref{tab-principios}, podemos perceber que eles garantem que a disponibilização de dados governamentais seja orientada de acordo a possibilitar a apropriação desses dados por parte dos cidadãos, que podem reutilizá-los na rede.

Um outro ativista que contribuiu também para a definição de dados governamentais foi o David Eaves \cite{3leis}. Em 30 de setembro de 2009, Eaves apresentou o painel \emph{Conference for Parliamentarians: Transparency in the Digital Era}, em um evento no Canadá, onde o objeto era debater e refletir sobre o novo paradigma que o mundo digital inaugura para o direito a informação. Como parte desse painel, Eaves apresentou três leis dos dados governamentais abertos, que seguem descritas no quadro abaixo:
\begin{table}[H]
\begin{center}
    \begin{tabular}{ | p{10cm} | }
    \hline
   Se o dado não pode ser encontrado e indexado na web, ele não existe.\\ \hline
   Se não estiver aberto e disponível em formato compreensível por a máquina, ele não pode ser reaproveitado.\\ \hline
   Se algum dispositivo legal não permitir sua replicação, ele não é útil.\\ \hline
    \end{tabular}
    \caption{As três leis dos dados abertos governamentais.}
    \label{tab-leis-lehman}
\end{center}
\end{table}
O conceito de dados governamentais abertos, portanto, se relaciona com um entendimento de que a forma como os governos disponibilizam suas informações permite que a inteligência coletiva crie melhores formas de trabalhar com elas do que os próprios governos poderiam fazer \cite{silva}.

Para que as organizações consigam publicar seus dados, o governo vem disponibilizando uma série de documentos que podem servir de guias para os próprios órgãos do governo, desenvolvedores ou qualquer interessado em abertura de dados. São alguns exemplos, entre outros:

\begin{itemize}
\item Manual dos Dados Abertos: Governo;
\item Manual dos Dados Abertos: Desenvolvedores;
\item Cartilha para desenvolvedores;
\item Folheto sobre Dados Abertos;
\item Cartilha Técnica para Publicação de Dados Abertos no Brasil;
\item Guia de Abertura de Dados;
\item Arquitetura Técnica Referencial para Abertura de Dados;
\item Guia de Dados Abertos (\emph{Open Knlowledge Foundation}).

\end{itemize}
Com esse arcabouço teórico e com a divulgação da importância de se publicar os dados de forma aberta pelo governo federal, espera-se que a população e o governo sejam beneficiados de várias formas, como por exemplo:

\begin{itemize}
\item Transparência e controle democrático;
\item Participação popular nas decisões das entidades que abriram dados;
\item Empoderamento dos cidadãos;
\item Melhores ou novos produtos e serviços privados;
\item Inovação através do reuso dos dados como informação;
\item Melhoria na eficiência e na efetividade de serviços governamentais;
\item Medição do impacto das políticas;
\item Conhecimento novo a partir da combinação de fontes de dados e padrões.

\end{itemize}















